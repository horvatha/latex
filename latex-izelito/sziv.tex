\documentclass[12pt]{article}
\usepackage{ucs}
\usepackage[utf8x]{inputenc}
\usepackage[english, german, magyar]{babel}
\usepackage{shapepar}
\usepackage{picinpar}
\usepackage{graphicx}
\usepackage{cwpuzzle}
\usepackage{color}

\setlength{\parindent}{0pt}
\setlength{\parskip}{\baselineskip}

\begin{document}

\title{\LaTeX\ ízelítő}
\author{Horváth Árpád}
\maketitle


%\section{Ide jöhet a saját fejezetem}
% Segít a latex-rovid.pdf fájl.

\section{Mi az a \LaTeX?}
\parbox{0.45\textwidth}{Egy dokumentumleíró nyelv, mely sokmindenben a
HTML-hez hasonlít, de végcélja egy nyomtatott dokumentum készítése.
Vannak hozzá olyan szövegszerkesztők, amelyeknél a végeredményt rögtön
látom, de mivel rengeteg különböző (akár házilag írt) csomagot
használhatunk vele, ezért lehetetlen, hogy mindet ismerje.}
%
\rotatebox[origin=c]{30}{
\parbox{.55\textwidth}{
Példák dokumentumformátumokra:\\[1ex]
\begin{tabular}{|c|c|}
\hline
WYSIWYG		&forrásként is szerkeszthető\\
\hline
MSWord		&\TeX\\
OpenOffice.org	&\LaTeX\\
AbiWord		&HTML\\
KOffice		&XML\\
\hline
\end{tabular}
}}%parbox és rotatebox vége.

%Az elforgatáshoz a graphicx csomag kell.


\newpage
\section{Miért szeretem a \LaTeX-et}
%%%
%%% shapepar
%%%
{\color{red}
\begin{center}
%\parbox{5cm}{
\heartpar
{ Ebben a rövid előadásban szeretném megmutatni, hogy miért érdemes \LaTeX-et
használni. Hogy pár dologban ugyan kellemetlenebb mint a ,,rögtön 
látom'' szövegszerkesztők, azonban több dologgal kárpótol ezért.
Pár dolog, ami teszik benne, hogy saját maga foglalkozik az elválasztással,
és a kinézettel,
nekem csak a dokumentum tartalmi részére kell összpontosítanom.
Ügyesen sorszámozza az ábrákat, fejezeteket, táblázatokat.
A \LaTeX\ fájl kicsi, a belőle készülő dvi, PDF vagy PostScript fájlok
viszont teljesen ugyanazt mutatják egyik gépen (PC+Windows), mint egy másikon
(SUN+UNIX). Ráadásul sok program áll rendelkezésemre teljesen ingyen az 
ábrakészítéstől (xfig) elkezdve a(z akár kétváltozós) grafikonok
elkészítésén (gnuplot, pylab) át az egyes szakterületek munkáját megkönnyítő
\LaTeX-csomagokig (különleges diagrammok, ábrák betűinek szöveghez
igazítása, kémiai képletek, kotta, keresztrejtvény!).
  S bár a képletek szerkesztése némi tanulást igényel, ez az idő
megtérül annak, akinek gyakran kell.
Egész jó prezentációk is készíthetőek a beamer csomaggal.
Végül az is kellemes, hogy a képletekkel teli
munkákat HTML-be át tudom alakítani. }%}
\end{center}
}%Idáig piros

\newpage
\section{Elválasztás}
Az elválasztást, a magyar dátumformát, ábraaláírást (1. ábra), a dokumentum részeinek neveit
(Tartalomjegyzék\dots) a \texttt{babel} csomag csinálja az utolsónak
felsorolt nyelv alapján.

\parbox{7mm}{Ezt a magyar helyesírás szerint választja el, a következőt viszont az angol szerint.}\qquad\qquad
\foreignlanguage{english}{\parbox{7mm}{In english aren't very long words. integrated circuit, pamphlet, pantomime, steadfastness, unjustifiable}}\qquad\qquad
\foreignlanguage{german}{\parbox{7mm}{Es ist Deutsch. Arbeitslosenunterstützung, Kraftwerk, Tonbandger\"{a}t.}}

Ha mégsem lenne jó az elválasztás (ez a magyar résznél hamar kitűnik),
akkor az elején valószínűleg jelzi a pdflatex, hogy a német és magyar
elválasztást nem tudta betölteni.  Ubuntu és Debian alatt ekkor a
texlive-lang-hungarian és texlive-lang-german csomagokat kell betölteni.

\newpage
%%% picinpar
%%%
\section{A picinpar csomag}
A következő bekezdés példát mutat a picinpar csomag alkalmazására.

\begin{window}[1,c,{\fbox{ABLAK}},{}]
Ebben a bekezdésben az első sor után megjelenik egy ABLAK
,,ablakkerettel'' körülvéve.  Az ablak előtt kezdődött sorok az ablak
után folytatódnak.  Ábrák vagy táblázatok beágyazása ugyanígy
történik.  Az opcionális argumentum harmadik és negyedik eleme
egyetlen blokk kell hogy legyen. 
\end{window}

\newpage
\section{Keresztrejtvény}

(cwpuzzle csomag)

\begin{Puzzle}{6}{7}%
|{}	|{}	|{}	|{}	|[1]P	|{}	|.	
|{}	|{}	|{}	|{}	|R	|{}	|.	
|{}	|{}	|[2]H	|*	|O	|{}	|.	
|{}	|{}	|I	|*	|T	|{}	|.	
|[3]L	|C	|G	|*	|O	|{}	|.	
|H	|*	|[4]G	|E	|N	|F	|.	
|[5]C	|M	|S	|{}	|{}	|{}	|.	
\end{Puzzle}

\begin{PuzzleClues}{\textbf{Vizszintes: }}%
  \Clue{3.}{LCG}{A CERN számításai ezen futnak}%
  \Clue{4.}{GENF}{Ez a város a CERN székhelye}%
  \Clue{5.}{CMS}{A hadronütköztető legnagyobb tömegű detektora}%
\end{PuzzleClues}%
\begin{PuzzleClues}{\textbf{Függőleges: }}%
  \Clue{1.}{PROTON}{Atommagban van}%
  \Clue{2.}{HIGGS}{Ezt keressük még a Standard modellből}%
  \Clue{3.}{LHC}{Ez a nagy gyorsító indult 2008-ban}%
\end{PuzzleClues}%

\newpage
\tableofcontents


\end{document}
